\documentclass[a4paper]{article}
\usepackage[utf8]{inputenc}

\title{Differenzialrechnung Zusammenfassung}
\date{Oktober 2020 - Februar 2021}

\begin{document}

\maketitle
\tableofcontents

\section{Potenz-, Sinus- und Kosinusfunktion, Exponential- und Logarithmusfunktion}
1. Potenzregel 
$ f(x) = x^n, n \in \mathds{R} \longrightarrow f'(x) = n \cdot x^{n-1} $\\
Beispiele: 
$f(x) = x^5 \longrightarrow f'(x) = 5 \cdot x^4$ \\
$f(x) = \frac{1}{x^{3}} = x^{-3} \longrightarrow f'(x) = (-3) \cdot x^{-4} = -\frac{3}{x^{4}} $ \\
2. Sinus- und Kosinus Funktion \\
$f(x) = sin(x) \longrightarrow f'(x) = cos(x) $ \\
$f(x) = cos(x) \longrightarrow f'(x) = -sin(x) $ \\
3. Exponentialfunktion \\
$f(x) = e^x \longrightarrow f'(x)=e^x$ \\
4. Logarithmusfunktion \\
$ f(x) = ln(x) \longrightarrow f'(x) = \frac{1}{x}$
\section{Funktionen mit der Summen- und Faktorregel ableiten}
Zusammengesetzte Funktionen mit der Produkt-, Quotienten und Kettenregel ableiten. \\ 
1. Summen- und Faktorregel \\
$f(x) = k_1 \cdot f_1(x) + k_2 \cdot f_2(x) \longrightarrow f'(x) = k_1 \cdot f_1'(x) + k_2 \cdot f_2'(x) $ \\
Beispiele: \\
$f(x) = 5 \cdot x^3 + 4 \cdot x^2 \longrightarrow f'(x) = 5 \cdot 3 \cdot x^2 + 4 \cdot 2 \cdot x = 15 \cdot x^2 + 8 \cdot x$ \\
$f(x) = \frac{3}{x^2} - 6 \cdot \sqrt{x} \longrightarrow f'(x) = 3 \cdot (-2) \cdot x^{-3} - 6 \cdot \frac{1}{2} \cdot x^{-\frac{1}{2}} = - 6 \cdot x^{-3} - 3 \cdot x^{-\frac{1}{2}} = \frac{6}{x^3} - \frac{3}{\sqrt{x}}  $ \\
2. Produktregel \\
$f(x) = u(x) \cdot v(x) = u'(x) \cdot v(x) + u(x) \cdot v'(x) $ \\
Beispiele: \\ 
$f(x) = (2x+5) \cdot e^x \longrightarrow f'(x) = 2 \cdot e^x + (2x+5) \cdot e^x = (2x + 7) \cdot e^x $
$f(x) = (x^2 + 3x - 1) \cdot e^x \longrightarrow f'(x) = (2x+3) \cdot e^x + (x^2 + 3x -1) \cdot e^x = (x^2 + 5x +2) \cdot e^x $ \\ 
3. Falls Funktionsterm als Bruch geschrieben werden kann \textrightarrow  Quotientenregel: \\
$f(x) = \frac{u(x)}{v(x)} \longrightarrow f'(x) = \frac{u'(x) \cdot v(x) - u(x) \cdot v'(x)}{v^2(x)}$ \\ 
Produkt- und Kettenregel zusammengesetzt: \\
$f(x) = \frac{u(x)}{v(x)} = u(x) \cdot (v(x))^{-1} \longrightarrow 
f'(x) = u'(x) \cdot v(x)^{-1} + u(x) \cdot (-1) \cdot (v(x))^{-2} \cdot v'(x) = \frac{u'(x)}{v(x)} - \frac{u(x) \cdot v'(x)}{v^2(x)} = \frac{u'(x) \cdot v(x) - u(x) \cdot v'(x)}{v^2(x)} $ \\ 
4. Kettenregel \\
$(u\bullet v)'(x_0) = u'(v(x_0)) \cdot v'(x_0)$ \\ 
$f(x) = g(h(x)) \longrightarrow f'(x) = g'(h(x) \cdot h'(x))$
$f(x) = e^{-x^2} \longrightarrow f'(x) = e^{x \cdot ln(3)} \cdot ln(3) = 3^x \cdot ln(3) $
\section{Funktionen mit der Produkt-, Quotienten- und Kettenregel ableiten}

\end{document}
